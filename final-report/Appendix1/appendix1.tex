\chapter{Third-Party Code and Libraries}
\label{third-party}

\section{FundFind}
\subsection{Python Packages}
\subsubsection{Python Standard Distribution Packages}
|json| is used for parsing the JSON data format (e.g. the app configuration is in a JSON file, the API responds with JSON).

|sys| and |os| are Python packages for interacting with the Operating System. |sys| contains certain global pieces of information such as the running Python version number.

|logging| handles logging output such as warnings, errors or just debugging information to the command line and to files.

|uuid| handles the generation of UUID-format unique identifiers.

|UserDict|, when subclassed by an application class, gives that class most of the properties of a dictionary. All |fundfind.dao| objects have property since they inherit from 
\\|fundfind.dao.DomainObject| which in turn is a child of |UserDict|. This allows things like |account['department']|.

|httplib| is one of the myriad of Python libraries for handling HTTP connections. |requests| is another one, albeit much cleaner and easier to use and is used in the FundFind tests. |httplib| is used by certain |dao| methods for accessing the datastore which were not changed from the parent IDFind codebase and can probably be refactored out of use by making more extensive use of the |pyes| elasticsearch access library.

|datetime| handles time in Python.

|re| is the standard regular expressions Python module.

|unicodedata| lends a hand in handling Unicode characters to the |util| module when producing URL-friendly slugs of strings.

\subsubsection{Other Python Packages}

Flask (|flask| in imports) is the main back-end framework upon which the whole web application is based \cite{flask}. The |fundfind.dao| module imports |werkzeug|, which is the underlying web server component for Flask for help with generating user password hashes.

The Flask-Login extension (|flask.ext.login| in imports) was used for helping with user authentication \cite{flask-login}. It does not provide registration capabilities or creating profiles.

|pyes| was used for connecting to elasticsearch datastore \cite{pyes}.

|parsedatetime| \cite{parsedatetime} is a module which tries to guess the time and date format from a simple string.

\subsection{Non-Python Code}
Web user interface libraries - Bootstrap \cite{bootstrap}, jQuery \cite{jquery}, jQuery UI \cite{jquery-ui}, linkify \cite{linkify} (linkify is just a jQuery plugin which turns text on HTML pages that looks like a URL into an actual clickable URL).

Search page - facetview \cite{facetview}, developed as a jQuery plugin.


\section{Funding Harvest}
Funding Harvest uses the standard distribution Python packages |datetime|, |logging|, |json|, |sys|, |uuid|, |UserDict| and |httplib|, described above, for the same reasons (mostly |config| and |dao| modules, which are very similar to FundFind's). It also uses |copy|, which a standard module which allows deep copies of objects to be made (i.e. when copying complex objects the actual values are copied, not just pointers to other objects, no matter how complex the hierarchy).

It does use one third-party (not a Python standard) module - feedparser \cite{feedparser} for parsing RSS feeds.
