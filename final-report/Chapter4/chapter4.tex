\chapter{Implementation}

% The implementation should look at any issues you encountered as you tried to implement your design. During the work, you might have found that elements of your design were unnecessary or overly complex, perhaps third party libraries were available that simplified some of the functions that you intended to implement. If things were easier in some areas, then how did you adapt your project to take account of your findings?

% It is more likely that things were more complex than you first thought. In particular, were there any problems or difficulties that you found during implementation that you had to address? Did such problems simply delay you or were they more significant? Your implementation might well be described in the same chapter as Problems (see below).

% working within context of other projects, have to keep conventions in mind, although better ways of doing things always welcome
\section{User Interface}
\label{impl-ui}

\section{Implementing The Main Application}

\subsection{Faceted Search}
\label{impl-facetview}
% describe how searches can be shared just by copying the url of the search page

\subsection{The Final Features}

\subsubsection{Responsive Mobile Web Design}
\label{impl-mobile}

% the suggestions - initially decided to code as pure Javascript since all that is really needed is an AJAX call to the Gateway to Research API, which will happily return JSON. However, this breaks the commitment to providing all of FundFind's functionality via an API. Incidentally, Richard Jones from Cottage Labs LLP had written a Python client library in preparation for the GtR hackday in order to explore the GtR API. This made for an easy decision - FundFind was to use the Python library was used to access the GtR API. It would then expose the results as suggestions TODO

\subsection{The Dropped Features}
% TODO add all the dropped features from chapter 2, Initial Requirements here
The decision to drop some features came after some design and exploratory technical work had been done.

\subsubsection{Harvesting Funding Data From E-mail Digests}
\label{impl-email-parse}
% chap2 says It is far easier to process RSS feeds such as the EPSRC Open Calls RSS Feed \cite{epsrc-rss} than it was to process the EPSRC Funding Call E-mail Alerts, as discussed in \myref{impl-email-parse}.
%fundfind.email.harvester@gmail.com