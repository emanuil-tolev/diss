\chapter{Background and Objectives}

%This section should pick-up material from your progress report and enhance it based on the feedback and also your additional experience up to now.

%outline the research context or commercial context in which the work was carried out;
%state clearly the objectives of the work;
%review the relevant literature and any similar existing systems;

\section{The Project}
\label{intro-project}
This project is about learning to deal with the field of funding in modern scholarship\footnote{A note on ``scholarship'' - the word and its derivatives are used throughout this text to mean all types of academic work, due to ``science'' being sometimes perceived as exclusionary to the Arts and Humanities fields.} and exploring how it could be opened up, from the point of view of a software developer. It is set within the context of the Open Knowledge field and seeks to add to this field by exploring how scholarly funding can be made more transparent and easier to find for all interested parties.

The main vehicle of learning and main software output is an open-source web application named ``FundFind''. It's a simple crowdsourcing application which allows registered users to share information about funding opportunities under open\footnote{The Open Definition defines ``open'' well \cite{od}.} terms. Its source code is available at \cite{fundfind-src}. Access to submitted information includes a machine-friendly API (Application Programmable Interface).

One of the project features - getting funding data into the datastore - was developed as a separate technical project due to natural low coupling between the main application and this feature. It is called ``Funding Harvest''. Its source code repository is publicly available at \cite{fundingharvest-src}.

\section{Project Aims}
\label{intro-aims}
\begin{itemize}
 \item Learn more about the funding sector and identify user requirements within it. Results presented in \myref{audience} and more concretely in \myref{devprocess-requirements}. The data must be sufficient to enable the creation of a project roadmap for
    \begin{enumerate}[a)]
    \item future research work into the requirements of user groups which were not engaged during this project
    \item concrete development work which would enhance the software built during this project
    \end{enumerate}
  The foundations for this are laid in in \myref{future-work}.
  
 \item Identify how much software it is possible to build by essentially starting and seeing how far it goes in a controlled fashion. The author's own initial lack of domain knowledge about field and technical skill level mean than a fixed feature list is hardly possible. However the progression of features and the final ``shape'' of the FundFind software is discussed in \myref{devprocess-requirements}.
\end{itemize}

\section{Scholarship and Funding}
Scholars need money to conduct research. To be precise, research projects need \emph{additional} money (on top of the investigators' salaries) in order to cover material expenses, travel costs, hiring Research Assistants or providing studentships to PhD candidates and other expenses. 

In other words, established scholars need to look for funding for aims which may be hard to define, for amounts of money ranging from thousands to millions.

Postgraduate-level scholars (candidates, postgraduates, post-doctoral researchers) are also usually looking for a way to fund the next stage of their scholarly career. (Note: undergraduate-level funding was not considered in this project.)

The quintessential problem is that the information about available funding is published in many different places. Usually each funding organisation publishes its own calls for proposals and announces its own studentships on its own website(s) and mailing list(s).

Keeping up with these sources of information becomes quite difficult - so much so, that it takes a significant portion of research development officers' time. Worse still, this is a significant barrier for those looking into a career in academia - the fragmented information can make it difficult to build a mental picture of what the funding in one's chosen field ``looks like''. This leads to a feeling of opaqueness instead of transparency and being overwhelmed - instead of having a clear idea of what it takes to succeed in a chosen field. Sometimes it even leads to accusations of cronyism in distributing funding (e.g. \cite{cronyism1}).

Furthermore, if this process is that difficult for person working in a specific country, then it can easily be estimated that globalisation of scholarship and scholars' mobility is hindered as well.

These problems seem to stem from the fact that the distribution of scholarly funding seems to have evolved over time from a series of changes to more general money distribution in society. This is actually a well-known problem that the software development field has had to deal with - if multiple stakeholders with distinctive interests try to influence the development of a software system, changing its requirements over time, the results can be disastrous.

There is no easy way to track money - where it gets allocated to, what fields (or even topics) within scholarship get funded more than others. Up until November 2012, there was no publicly accessible \emph{centralised} detailed account of historic funding information in the United Kingdom. The Gateway to Research project being implemented by Research Councils UK finally gives a partial (RCUK-specific) overview of historical data.

However, there is still no mechanism of getting \emph{federated, present, current} data about scholarly funding.

Something which allows searching the (vastly) different, \emph{currently available} streams of funding within academia was thus identified as potentially useful by various people interested in the subject.

\subsection{The Value of Openness in Scholarship and Elsewhere}
The Free Software, Open Source and more recently, Open Knowledge (which Open Access is a part of) movements have all demonstrated that the value of information can be greatly multiplied simply by having more people access it, reuse it and even be creative with it (e.g. visualise or summarise). The Open Knowledge Foundation argues this \cite{okfn-vision}. Seminal works like ``The Cathedral and the Bazaar'' \cite{catb} have also argued this.

Governments seem to have grasped the benefits as well. The UK Government has mandated that all research funded with public money (which is a lot in the UK - almost everything through the seven Research Councils) should be published as Open Access by 2014 \cite{guardian-ukgov-oa2014}. It has also funded initiatives such as Open UK governmental data in the form of data.gov.uk \cite{open-uk-gov-data} and the recent Finch report \cite{guardian-finch} \cite{finch}.

The ``Open'' part of ``Open Access'' means more than just throwing the data out there in any form and format. The Open Definition \cite{od} has a lot to say about all aspects of Openness. However, principles \#1 ``Access'' and \#4 ``Absence of Technological Restriction'' are relevant, so it can be seen why the current practice of each funder publishing their own HTML pages is not good enough:

\begin{itemize}
 \item ``The work shall be available as a whole [...] ``As a whole'' prevents the limitation of access by indirect means, for example by only allowing access to a few items of a database at a time.''
 \item ``The work shall be available [...] in a convenient and modifiable form. [...] The work must be provided in such a form that there are no technological obstacles [...]. This can be achieved by the provision of the work in an open data format.''
\end{itemize}

In practice, when applied to scholarly output such as publications, these have been observed by the author to mean ``PDF-s are not going to cut it'', ``You must be able to data mine it''.

\emph{So why not have this for funding data?} Currently, separate HTML pages, weekly e-mails and (sometimes) RSS feeds are all that the Research Councils UK do to publish their funding information. RCUK, in particular, are a publicly funded organisation, and it could be argued that this data belongs to the public. In any case, convincing both public and private funders of the benefits of Open funding data is needed.

Private organisations are not far behind and may in fact be leading the way - one of the largest funders of science in the UK and around the world, the Wellcome Trust, has had a progressive and strict Open Access policy since 2006 \cite{wellcome-oa}.

\section{Audience and High-Level Requirements}
\label{audience}
The potential audience of this project was discovered to be quite varied. It seems multiple stakeholders stand to benefit from more transparent scholarly funding dissemination.

The requirements or potential uses that these users might want to put FundFind to are closely tied to the benefits that come with opening up funding data.

\subsection{Professionals Looking for Funding Group}
\label{audience-professionals}
\subsubsection{Accomplished Scholars}
\begin{itemize}
 \item \textbf{Who}: Academics - lecturers, professors
 \item \textbf{Aims}: Wish to fund future projects
 \item \textbf{Benefits from opening up funding data}: Easier access and search across funding institutions through tools which aggregate funding data; Transparency and accountability of their funding bodies.
 \item \textbf{Possible requirements from FundFind}: Aforementioned cross-funder search; Sharing opportunities with colleagues.
\end{itemize}

\subsubsection{Junior Academics}
\begin{itemize}
 \item \textbf{Who}: Postgraduate candidates, postgraduate students, academics in post-doctoral posts (e.g. Research Associates)
 \item \textbf{Aims}: Wish to find suitable positions to move on with their academic career, e.g. a postgraduate with a PhD will be looking for a post-doctoral position
 \item \textbf{Benefits from opening up funding data}: (More) easily find and compare funding sources
 \item \textbf{Possible requirements from FundFind}: Find funding. To quote a postgraduate candidate: ``I want to do a taught MSc in Astrophysics but I can't afford to self fund. Where can I go and who will pay me?''
 
 Share funding information with peers (e.g. if a candidate for a Computer Vision PhD finds a good PhD studentship call for applications in Bioinformatics, they may well want to forward it to a colleague).
\end{itemize}

\subsubsection{Research Development Officers}
\label{audience-research-officers}
\begin{itemize}
 \item \textbf{Who}: Professionals who find more funding opportunities for scholars
 \item \textbf{Aims}: Wish to record the opportunities they find somewhere and share them with scholars. Also may want to access a list of all opportunities \emph{they} have found over time, so that they can report to their line managers more easily.
 \item \textbf{Benefits from opening up funding data}: Not as many as scholars, unless they are happy to share the opportunities they have found with research officers from other institutions (which they do not always want). However, a system which allows sharing of recorded opportunities with a select subset of users (such as all within the officer's institution) would be welcome as this can allow them to record a funding opportunity once, share it with ``their'' scholars and include it in reports.
 \item \textbf{Possible requirements from FundFind}: Find funding. Share funding with subset of users. See information submitted per user (themselves).
\end{itemize}

\subsection{Funders Group}
Those looking to advertise funding opportunities - Funding Stream and Programme managers at various institutions.

The needs of this group are estimated since funders were not engaged as users at this point in FundFind's development due to time constraints.

\subsubsection{Research Councils in the UK}

\begin{itemize}
 \item \textbf{Who}: Main way of allocating \emph{public} scholarly funding in the United Kingdom.
 \item \textbf{Aims}: Wish to attract the best scholars in their own field, but also look to establish new cross-disciplinary  projects.
 \item \textbf{Benefits from opening up funding data}: Already committed to opening up historical funding data (e.g. Gateway to Research project \cite{rcuk-on-gtr}), would like to advertise current funding data more widely. The benefits are widening impact and increasing the inter- and intra-field interconnectedness of scholarship (scholars can learn about interesting funded projects and people that they didn't know about).
 \item \textbf{Possible requirements from FundFind}: Mostly just search and upload of information. In particular: search for submitted items related to their institution. Bulk upload of opportunities. Control over the ``funder'' profile related to them.
\end{itemize}

\subsubsection{Jisc}

\begin{itemize}
 \item \textbf{Who}: Used to be known as ``Joint Information Systems Committee'', but have decided to rebrand \cite{jisc-rebrand}. The old name of this public organisation reflects their purpose quite well - helping the adoption of digital technology in research and teaching in the UK.
 \item \textbf{Aims}: Wish to attract talent to work on scholarly infrastructure in the UK, i.e. give their aims and projects more exposure. Scholars rarely choose scholarship infrastructure as their main field.
 \item \textbf{Benefits from opening up funding data}: Have always wanted wider reach, as well as vigorously supported Open Data. Have a lot of funding opportunities, both minor and larger ones. Would like to advertise them better.
 \item \textbf{Possible requirements from FundFind}:  The same as RCUK. In addition, however, they might wish to actually use FundFind's data in other projects - in other words, they might be an interested API consumer.
\end{itemize}

\subsubsection{Charities, Trusts and Others}

\begin{itemize}
 \item \textbf{Who}: Various charitable Foundations, Institutes, Endowments and so on which fund scholarly activities (e.g. the Wellcome Trust \cite{wellcome-trust}, Shuttleworth Foundation \cite{shuttleworth-foundation}, Knight Foundation \cite{knight-foundation}).
 \item \textbf{Aims}: Simple - want the best scholars competing for their money (at least the portion dedicated to conventional research).
 \item \textbf{Benefits from opening up funding data}: Need to reach more scholars to build up an ever-improving pool of applicants (in terms of quality). Also, as private institutions, would like to demonstrate greater transparency with their funding dissemination and allocation as part of their efforts to prove they are doing it effectively.
 \item \textbf{Possible requirements from FundFind}: Same as RCUK.
\end{itemize}

\subsubsection{Commercial Organisations}

\begin{itemize}
 \item \textbf{Who}: All companies which perform R\&D activities. Famous UK examples include Rolls-Royce \cite{rolls-royce} and IBM \cite{ibm}.
 \item \textbf{Aims}: The ones with large enough R\&D budgets to get involved with scholars usually invest in whole laboratories and centres. When they do share effort and money with research institutions, they do have to search for the ``best scholars'' to collaborate with. The smaller ones may have a problem however - they may only need external help with R\&D intermittently and thus may not have a ``constant'' audience of scholars following their every move since they would not be allocating funding all the time.
 \item \textbf{Benefits from opening up funding data}: Greater transparency of funding allocation, potentially greater audience if a federated scholarship funding source like a popular FundFind instance does come into existence at some point.
 \item \textbf{Possible requirements from FundFind}: Same as RCUK.
\end{itemize}

\subsection{Analyst Group}
\label{audience-analyst}
Those looking to analyse scholarly funding - how it is allocated and/or spent.

A hackday in the middle of March which focused on the Gateway to Research project provided insight into the usefulness of certain initially planned features and inspired new ideas. The main audience of the hackday was this group. The GtR is an attempt to get the seven UK Research councils to publish historical funding data - who got funded, grant codes, who worked with whom, any related publications \cite{gtr}, etcetera via a web interface and a RESTful API. GtR is quite similar to FundFind, except that it relates to \emph{historical}, not current funding information, and of course contains a lot more information than FundFind is currently scoped to hold.

\begin{itemize}
 \item \textbf{Who}: Software developers, journalists and others interested in Open Knowledge and visualising data for the purposes of transparency or advancing the digital economy \cite{okfn-vision} The author is included in this group. Other interested parties are analysts, bloggers and the general public.
 \item \textbf{Aims}: Want transparency and accountability. May want to visualise how scholarship money is being spent or otherwise help society engage with scholarly spending.
 \item \textbf{Benefits from opening up funding data}: Open Data essential for re-use for their aims. Usually do not have first hand information about what happens behind closed doors when funding is distributed and used by universities. Think that this is strange, since often both funder and fundee are public sector organisations / employees.
 \item \textbf{Possible requirements from FundFind}: 
\end{itemize}

There is another group which belongs here - politicians. They may wish to demonstrate accountability and good decision making. Despite the fact that they are not usually known to readily embrace new technological solutions, they do do it when the advantages (at least from a Public Relations perspective, if no other) become clear. For example, the UK government more or less suddenly embraced Open Access for all publicly funded research \cite{guardian-ukgov-oa2014}.

\subsection{Focusing on Certain Audience Groups}
\label{focus-groups}
Satisfying the requirements of all user groups is more suitable for a larger, better-resourced project than this one.

This project will focus on the first group of users'' meaning \myref{audience-professionals}. The main reasons are still ``time constraints and ease of access to such users''. However, the \myref{audience-analyst} was also taken into account. This was partly because the success of a technological project just has to include other developers, especially when it is an open-source project. On the other hand, the author is part of this group and day-to-day interactions create unavoidable bias, especially when they include learning (how to use technologies relevant to FundFind) and feedback on the very project being developed.

As stated previously in the Progress Report, FundFind does have an open machine-ready to communicate with other software - an Application Programmable Interface (API). In terms of concrete people within this group, Cottage Labs LLP \cite{cl} still find value in the project and will probably help to take it further after the current development as a piece of course work ends.

In addition to the The Open Knowledge Foundation \cite{okfn-vision}, to whom the project will be presented after it is finished and feedback will be asked for, the Gateway to Research technical team will also be contacted about possibly building a stronger relationship with FundFind. They might be interested in its use of the data provided by their API to make suggestions of interesting projects (to look at) to scholars. Furthermore, they handle \emph{historical} funding data, whereas FundFind handles \emph{current} funding data. Convincing the Research Councils UK to provide the same uniform interface to their \emph{current} funding data as they are to their \emph{historical} data with the GtR project would be a great boon to the Open Knowledge movement.

\section{Existing Works}
\label{existing-works}
There is no single piece or combination of pieces of software which enables the identified audience to do what this project would allow them to do upon completion, to the best of the author's knowledge.

Related Open Knowledge projects are still as relevant as at the beginning of the project, in particular the Open Knowledge Foundation ones at \cite{okfn-labs} and \cite{okfn-github}. \myref{design-rationale} notes individual projects which informed FundFind's technical design in some way.

The FundFind codebase was initially based entirely on the IDFind project \cite{idfind}, which is available under the MIT license \cite{idfind-src}. Cottage Labs LLP and the author started it during the DevXS 2011 conference \cite{devxs}. IDFind was a good place to start, since the project's context (Open Knowledge, interested parties) is similar, so a similar set up and technologies could be used to accelerate the initial development.

\section{Licensing}

FundFind is licensed under the MIT license, like its parent IDFind. This is a permissive license which should facilitate the widest possible use and reuse of code. Moreover, there is little reason to place particular restrictions since the target Open Knowledge / Funding fields are narrow enough and there isn't really any open-source software which sits at the intersection of the two.

\subsection{Contributions}

Contributions through FundFind's user interface (crowdsourcing) are licensed under the Creative Commons Attribution (CC-BY) \cite{cc-by-3} license, version 3.0 (latest at time of writing). This was chosen after considering a multitude of other licenses. In the end, the new Budapest Open Access Initiative recommendations and the Open Definition \cite{od} recommending CC-BY informed the choice.

This license only applies to publicly visible (shared with the world) submissions. For now, it is not possible to share funding information only with certain users or groups - therefore, the license applies to all crowdsourced contributions. Proper access control was too costly to implement in light of more important / core features which would enable the system to at least act as a prototype for completely open sharing, before moving on to the more complex role of a virtual shelf of funding opportunities.

The contributions license does not apply to data which has been harvested from data sources (like funders' websites or RSS feeds) automatically. The owners of the data would have to give written consent for Funding Harvest to harvest their data and publish it as CC-BY (or, better and more likely - publish it as CC-BY themselves). Funding Harvest is only a tool which, in combination with FundFind, is meant to showcase the benefits of having funding data open, thus convincing data sources to publish data more openly. It is not a tool to infringe on the copyrights of others, although it can be used as such. This is true of any software which does page scraping or even just consumes and records RSS feeds.

Most records in FundFind's datastore denote the license they can be used under. If a record does not have a |license| field (user accounts do not, for example), then the default (restrictive) copyright rules apply to that data.