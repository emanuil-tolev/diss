\chapter{Evaluation}

%Such material is regarded as an important part of the dissertation; it should demonstrate that you are capable not only of carrying out a piece of work but also of thinking critically about how you did it and how you might have done it better. This is seen as an important part of an honours degree. 

%There will be good things and room for improvement with any project. As you write this section, identify and discuss the parts of the work that went well and also consider ways in which the work could be improved. 

%The critical evaluation can sometimes be the weakest aspect of most project dissertations. We will discuss this in a future lecture and there are some additional points raised on the project website. 

\section{Approaching the field of scholarly funding}
\label{eval-difficult-field}
% amorphous, ill-defined
% possible to achieve automation and great time/cost savings but will need consensus from funders
% fundees will follow, but funders want to be careful when making changes - not drive away good researchers who can do the job
% FundFind can be developed further to facilitate this - show funders the benefits of agreeing on a common representation of their data, show fundees the benefits of being able to search transparently across organisations
% FIXME
Somewhere in February, the 4th (out of 6) month of the project, I realised that this domain of knowledge, scholarly funding, was going to be a tougher nut to crack than I had thought.

decided late that the project was going to do requirements gathering as a formalised goal - inspired both by G4HE (Cottage Labs project) and difficulties in understanding the field \S\ref{eval-difficult-field}. These only became clear once the technical work started in earnest and it turned out it was in fact both a pretty big field, and one that was ripe with problems and conflicting interests.

TODO FIXME EXPAND THIS HERE INTO A SUBSECTION ABOUT WHAT LEARNED ABOUT FIELD. Regardless of personal interest, it takes time to develop holistic understanding of a field like this. One analogy would be another human field of endeavour the author has a personal interest in - saving and re-homing homeless animals. Ripe with conflicting interests with a troubled history of ad-hoc evolution resulting from the intersection of these interests. FIXME TODO EXPAND HERE exactly what are the problems? Politics, money and the sincere wish to save every animal encountered, the good intentions sometimes resulting in bad outcomes 
such as houses full of tens of animals which cannot be taken care of.

% FIXME
Relating information about the field and its problems to people outside the target audience and even across target user groups turned out to be much more difficult than expected. TODO more here?

\subsection{Condensing the insights gained}
Part of the point of the project was to see how the funding field worked. And this went fine, I now know a lot more about it than when I started. However, sharing this information seems really difficult. It feels like my sample was way too small and that any representation I choose for it will just be laughably bad. In reality, there is clearly a place in the world for a ``how does scholarly funding work for beginners, from a beginner'' collection of information, but presenting this appropriately seems incredibly difficult.

All the insights gained were attained through interviews. However, it was just my opinion of the content of these meetings that was actually going to make it into the final output. This can feel quite daunting - as a developer, I prefer to publish the data and let those whose work I support actually ``deal'' with it. However, in this case, I \emph{am} the researcher, this mysteriously skilled person who somehow knows what to do with this data. Others do not want to know about conversations I have had, they want knowledge - even understanding. My understanding. % TODO expand this

%However, I have done similar things before - I talked about what Open Knowledge is while I was still figuring it out, so I decided late on that one of the outputs of this project was to be a blog post on the topic. My previous attempt to tackle a new field is published on the Cottage Labs website, and we do publish thoughts which concern the Higher Education field, so this seemed like a good place to publish it \cite{fundfind-blog-post}.

\section{Collecting Information From Disparate Sources - Technological Suitability}
% Were the design decisions correct?
% Could a more suitable set of tools have been chosen?
% design decisions - evaluated
% better tools, better decisions possible?

\section{User Needs}
\subsection{Identifying Requirements}
% Were the requirements correctly identified? 

\subsection{Meeting Needs}
% How well did the software meet the needs of those who were expecting to use it?
% not about features, but data

\subsubsection{Those Looking To Analyse Scholarly Funding}

\subsubsection{Exploring Scholarly Funding}
Within the group of people with analytical needs, there was a sub-group of developers which inspired and was part of the reason the project was conceived - Cottage Labs LLP, the Open Knowledge Foundation and the author. It was always the project's intention to learn about scholarly funding specifically so that these people could have a better idea of how to meet the Higher Education sector's needs.
% How well were any other project aims achieved?
% learn more about it
% personally, Cottage Labs, OKFN, Open Knowledge movement

\section{In Retrospect}
% If you were starting again, what would you do differently?

\section{Future Work}
\label{future-work}
% what
% why not included in this project's scope
% why is it interesting

% talk about all dropped features? group them in what ways they could be used? or maybe make it feature-centric since some of them can be used in different ways

\subsubsection{E-mail Alerts}
This is a good feature and it's great that it was identified, especially since the author was not previously aware of the current scholars' workflow.

%research officers' reports \myref{audience-research-officers}).
\label{future-research-officers-reports}

\subsubsection{Exporting The Results Of Searches}
Nonetheless, research development officers' reports mentioned in \myref{audience-research-officers} could benefit from this feature, so it was added to Future Work, \myref{future-work}.

\section{Learning}
% LaTeX skills - not in industry, but if ever return to academia, lots of useful learned. Wordcount script, macros, large documents, divide focus using chapter files.
% Open Knowledge field
% funding field
% Python for web applications, specifically in the way it's used by future employer