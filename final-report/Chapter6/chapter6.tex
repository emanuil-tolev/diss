\chapter{Evaluation}

%Such material is regarded as an important part of the dissertation; it should demonstrate that you are capable not only of carrying out a piece of work but also of thinking critically about how you did it and how you might have done it better. This is seen as an important part of an honours degree. 

%There will be good things and room for improvement with any project. As you write this section, identify and discuss the parts of the work that went well and also consider ways in which the work could be improved. 

%The critical evaluation can sometimes be the weakest aspect of most project dissertations. We will discuss this in a future lecture and there are some additional points raised on the project website. 

\section{Approaching the field of scholarly funding}

% amorphous, ill-defined
% possible to achieve automation and great time/cost savings but will need consensus from funders
% fundees will follow, but funders want to be careful when making changes - not drive away good researchers who can do the job
% FundFind can be developed further to facilitate this - show funders the benefits of agreeing on a common representation of their data, show fundees the benefits of being able to search transparently across organisations

\section{Whose requirements?}
% Were the requirements correctly identified? 

\section{Collecting information from disparate sources - technological suitability}
% Were the design decisions correct?
% Could a more suitable set of tools have been chosen?
% design decisions - evaluated
% better tools, better decisions possible?

\section{User needs}
% How well did the software meet the needs of those who were expecting to use it?
% not about features, but data

\section{Exploring scholarly funding}
% How well were any other project aims achieved?
% learn more about it
% personally, Cottage Labs, OKFN, Open Knowledge movement

\section{In retrospect}
% If you were starting again, what would you do differently?